\documentclass[10pt,a4paper,openright]{article}
\usepackage{amsfonts}
\usepackage{float}
\usepackage[utf8]{inputenc}
\usepackage[T1]{fontenc}
\usepackage{graphicx}
\usepackage{amsmath,amssymb,verbatim}
\textheight=240mm
\textwidth=170mm
\topmargin=-15mm
\oddsidemargin=-6mm
\evensidemargin=\oddsidemargin
\newcommand\tab[1][1cm]{\hspace*{#1}}
\usepackage{hyperref}
\newcommand{\norm}[1]{\left\lVert#1\right\rVert}

\begin{document}
	\begin{flushleft}
		\large Name: Matouš Dzivjak\\
		\large Date: 7.12.2018\\
	\end{flushleft}
\begin{center}
	\huge Quadric
\end{center}
\section{Zadání}
\begin{verbatim}
https://cw.fel.cvut.cz/wiki/courses/b33opt/cviceni/domaci_ulohy/quadric/start
\end{verbatim}


\section{Řešení}

\subsection{Pokus č. 1}
Úlohu budeme řešit jako minimalizační úlohu vzdálenosti bodu $x$ od bodu $a$
za podmínky $x^TQx = 1$. Tedy využijeme Lagrangeových multiplikátorů. Napadnout nás mohlo 
například i lineární programování, ale omezující podmínka je ve tvaru rovnice nikoliv
nerovnice.

Podmínku $x^TQx = 1$ můžeme přepsat jako $x^TQx - 1 = 0$.
Dostáváme tedy dvě rovnice:

\begin{center}
	$\norm{x - a} = 0$

	\vspace{.1cm}

	$x^TQx - 1 = 0$
\end{center}

Vytvoříme Lagrangeovu funkci:

\begin{center}
	$\mathcal{L}(x, a, \lambda) = \norm{x - a} + \lambda (x^TQx - 1)$
\end{center}

To jsem udělal v matlabu. Nejdříve připravil matici symbolických proměnných $x$ 
a symbolickou proměnnou $\lambda$. Pak jsem rovnici $\mathcal{L}$ zderivoval podle
všech těchto proměnných $(x_{1}, x_{2}, ..., x_{n}, \lambda)$ a položil všechny
tyto rovnice rovny 0. Pak jsem použil matlabovskou funkci $solve$ pro proměnné
$x_{1}, x_{2}, ..., x_{n}$ tedy prvky matice $x$ a...

...nedostalo se mi žádného řešení. Vydal jsem se špatným směrem.
Řešení tedy bude nejspíše spočívat v některé z numerických metod.

\subsection{Pokus č. 2}
Vyžijeme symetričnosti matice. Vememe její spektrální rozklad:

\begin{center}
	$Q = V \Lambda V^T$
\end{center}

Kde $V$ je ortogonální matice s vlastními vektory ve sloupcích.
A $\Lambda$ je diagonální matice vlastních čísel.
Poté dostáváme:

\begin{center}
	$x^T Q x = x^T V \Lambda V^T x$
\end{center}

Můžeme tedy zadefinovat matici změny báze a udělat bijekci:

\begin{center}
	$y = V^Tx \iff Vy=x$
\end{center}

Úlohu následně vyřešíme pro $y$ a najdeme jednoznačně $x$ tímto inverzním zobrazením.
Tím se nám úloha mění na:

\begin{center}
	$\min\limits_{\forall y \in \mathbb{R}^ n} \norm{a - y}^2$ za p. $y^T \Lambda y = \sum^{n}_{i=1}(\lambda_i y_i^2) = 1$
\end{center}

Řešení lze nalézt pomocí Lagrangeových multiplikátorů:

\begin{center}
	$\mathcal{L}(y, \gamma) = \norm{a - y}^2 + \gamma (\sum^{n}_{i=1} (\lambda_i y_i^2) - 1)$
	
	\vspace{.2cm}
	
	$\frac{\partial \mathcal{L}(y, \gamma)}{\partial y_i} = -2(a_i - y_i + \gamma \lambda_i y_i)$

	\vspace{.2cm}

	$\frac{\partial \mathcal{L}(y, \gamma)}{\partial \gamma} = \sum^{n}_{i=1}(\lambda_i y_i^2)-1$
\end{center}

Kde $\lambda$ jsou diagonální prvky matice $\Lambda$ a $a_i$ jsou prvky vektoru $a$, tedy bodu,
ke kterému hledáme nejbližší bod na kvadrice.

Jak již jsme si ověřili v přechozím pokusu Laplacian nejde minimalizovat analyticky,
budeme ho tedy minimalizovat numericky - Newtonovo metodou.

\end{document}